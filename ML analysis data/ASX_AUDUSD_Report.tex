
\documentclass[11pt,a4paper]{article}
\usepackage[margin=1.5cm]{geometry}
\usepackage{graphicx}
\usepackage{amsmath}
\usepackage{hyperref}
\usepackage{float}
\usepackage{caption}
\usepackage{fancyhdr}
\pagestyle{fancy}
\fancyhead{}
\fancyfoot[C]{\thepage}

\title{Forecasting ASX and AUD/USD Returns Using Entanglement and Hedge Ratios}
\author{[Your Name] \\ Student ID: [Your Student ID]}
\date{}

\begin{document}

\maketitle

\section{Introduction}
Being from the finance field, having always found markets as some very fascinating realms of story. Among these stories, the US dollar (USD) stands out as the most-commonly traded currency, usually accepted as the chief reserve currency bought by central banks. Thus, the dollar’s status as a primary reserve currency comes from both high market liquidity and alignment with Australia's trading and reserve management needs (RBA, 2023). 

Against such a background, the research project aims to determine the relationship between the Australian Securities Exchange (ASX200) and the Australian dollar (AUD/USD), focusing on their entanglement in the markets and their hedge ratios. The financial nexus of Australian equities and exchange rate movements becomes an important concern in portfolio risk management as market participants face exposure to both sources of volatility.

The study seeks to ascertain whether dynamic hedge ratios and statistical entanglement, via Granger causality, can be used to anticipate next-day returns for the ASX200 and AUD/USD. Through the use of machine learning models, we investigate whether factoring in econometric constructs such as the minimum variance hedge ratio (MVHR) and Granger-causal features yield any improvements in forecasts.

Data sources include Yahoo Finance historical prices for ASX200 and AUD/USD, from which over 5000 daily observations spanning the period form 01/01/2001 to 20/05/2025  were collected.

\section{Literature Review}
\textbf{1. Hedging Behavior and Exposure Management} \\
Yang (2000) employed OLS, VAR, ECM, and VEC-GARCH to estimate hedge ratios in Australian futures. The RBA (2023) reported Australian firms hedged only \~35\% of FX exposures.

\textbf{2. Nonlinear Interactions and Granger Causality Detection} \\
Lu et al. (2023) proposed a machine learning system for nonlinear Granger causality. Lin et al. (2024) developed Kolmogorov-Arnold Networks (KANs) to improve interpretability.

\textbf{3. Machine Learning in Hedging and Entanglement Modeling} \\
Bao et al. (2017) showed that LSTM outperforms econometric models in time series prediction. Chong and Tan (2010) found exchange rates Granger-cause equity returns in Asia-Pacific.

\section{Method}
We predicted next-day log returns using lagged returns, rolling statistics, Granger test scores, and MVHR coefficients.

\textbf{Model Pipeline}:
\begin{itemize}
\item Feature engineering: log returns, rolling correlations, hedge ratios
\item Models: Ridge, SVR, Random Forest, Gradient Boosting
\item Validation: Nested CV (5 outer, 3 inner)
\item Metrics: RMSE, MAE, R\textsuperscript{2}
\end{itemize}

\section{Results and Discussion}
\textbf{Model Comparison} \\
GBM outperformed Ridge in both tasks, improving R\textsuperscript{2} by over 10\% in AUD/USD prediction.

\textbf{Granger Causality} \\
AUD/USD $\rightarrow$ ASX: F = 177.64 (p < 0.001); significant at all lags \\
ASX $\rightarrow$ AUD/USD: F = 11.70 (p < 0.001); weaker but present

\textbf{MVHR Regression} \\
$\beta = 0.3295$, $R^2 = 0.0606$

\textbf{Learning Curves} \\
GBM showed positive generalization trend, Ridge underfit quickly.

\section{Conclusion}
Tree-based ML models successfully leveraged statistical entanglement and hedge ratios to predict short-term returns. Granger causality was strong from AUD/USD to ASX. MVHR was only modestly effective. Further work may include macro variables and deep learning.

\section*{References}
\begin{itemize}
\item Bao, W., Yue, J., \& Rao, Y. (2017). A deep learning framework... \textit{PLOS ONE}, 12(7), e0180944.
\item Chong, T. T. L., \& Tan, K. H. (2010). ... \textit{Economics Bulletin}, 30(2), 1246--1254.
\item Lin, C., Wang, Y., \& Wang, Z. (2024). ... \textit{arXiv:2401.00940}.
\item Lu, Y., Wang, Y., Li, L., \& Wu, J. (2023). ... \textit{AAAI Conf. on AI}.
\item Reserve Bank of Australia (2023). \url{https://www.rba.gov.au}
\item Yang, W. (2000). M-GARCH Hedge Ratios... Griffith University.
\end{itemize}

\end{document}
